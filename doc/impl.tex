\newpage
\section{Implementazione}
%devono contenere: descrizione testuale, codice, l'output(solo per query/procedure)
\subsection{Query} %almeno 6 query, di cui alcune possono essere procedure
\begin{lstlisting}[title=Operazione 7, style=mysqlStyle] 
-- stampa codice, nome e quantita di tutti i prodotti che sono cibi e che scadono entro il 28 febbraio 2019, che inoltre iniziano con la R, ordinati in maniera decrescente --

CREATE VIEW CibiInScadenza as
SELECT DISTINCT C.CodiceProdotto, C.NomeProdotto, C.Quantita
FROM prodotto AS C
WHERE C.Categoria = 'Cibo' AND DataScadenza >= '2019-02-01' AND DataScadenza <= '2019-02-28' AND C.NomeProdotto LIKE "R%"
ORDER BY DataScadenza DESC;
\end{lstlisting}

\begin{lstlisting}[basicstyle=\ttfamily\footnotesize]
+----------------+--------------+----------+  
| CodiceProdotto | NomeProdotto | Quantita |  
+----------------+--------------+----------+ 
|           1400 | Ricotta      |        3 |  
|           1900 | Radicchio    |        4 |  
+----------------+--------------+----------+  
\end{lstlisting}

\begin{lstlisting}[title=Operazione 8, style=mysqlStyle]
-- stampa il codice, il nome, la marca e l'annata di tutti i vini con annata risalente a massimo il 2008 e prezzo inferiore a 20 euro, ordinati per prezzo crescente --

CREATE VIEW viniEconomici as
SELECT DISTINCT V.CodiceProdotto, V.NomeProdotto, V.Marca, V.Annata
FROM prodotto AS V
WHERE V.Categoria='Bevanda' AND V.Tipo='Vino' AND V.Annata <= 2008 AND V.Prezzo < 20 AND V.Quantita > 0
ORDER BY V.Prezzo;
\end{lstlisting}
\begin{lstlisting}[basicstyle=\ttfamily\footnotesize]
+----------------+--------------+---------------------+--------+  
| CodiceProdotto | NomeProdotto | Marca               | Annata |  
+----------------+--------------+---------------------+--------+  
|           1616 | Valpolicella | Terre di Terrarossa |   2007 |  
+----------------+--------------+---------------------+--------+  
\end{lstlisting}
\begin{lstlisting}[title=Operazione 9, style=mysqlStyle]
-- stampa il numero dei tavoli liberi nel giorno 20/02/2019 che non hanno 3 e 5 coperti --

CREATE VIEW tavoli3o5Coperti as
SELECT NumeroTavolo AS num
FROM tavolo AS T JOIN prenotazione AS P ON T.NumeroTavolo = P.NumeroTavolo
WHERE T.Coperti = 3 OR T.Coperti = 5 AND P.Giorno='2019-02-20'

CREATE VIEW tavoliLiberi as
SELECT T.NumeroTavolo
FROM tavolo AS T
WHERE T.NumeroTavolo NOT IN (SELECT num FROM tavoli3o5Coperti) 
\end{lstlisting}
\begin{lstlisting}[basicstyle=\ttfamily\footnotesize]
+--------------+  
| NumeroTavolo |  
+--------------+  
|            2 |  
|            3 |  
|            6 |  
|            8 |  
|            9 |  
|           11 |  
|           12 |  
|           13 |  
|           15 |  
|           16 |  
|           18 |  
|           19 |  
+--------------+  
\end{lstlisting}
\begin{lstlisting}[title=Operazione 10, style=mysqlStyle]
-- stampa nome e cognome di tutti i dipendenti di turno il martedi, raggruppati per ruolo il cui stipendio medio e' maggiore di 2000 --

CREATE VIEW dipendentiDiTurno as
SELECT DISTINCT P.Nome, P.Cognome
FROM personale AS P, svolgimento_turno AS T, turno AS Tr
WHERE T.CodiceFiscale=P.CodiceFiscale AND Tr.Giorno = 'Martedi' AND Tr.DataTurno=T.DataTurno
GROUP BY P.Ruolo
HAVING P.Stipendio >= 2000;
\end{lstlisting}
\begin{lstlisting}[basicstyle=\ttfamily\footnotesize]
+---------+------------+  
| Nome    | Cognome    |  
+---------+------------+  
| Fabio   | Marchi     |  
| Gianni  | Banzato    |  
| Maria   | Franceschi |  
| Massimo | Bertocco   |  
+---------+------------+  
\end{lstlisting}
\subsubsection{Procedure}
\begin{lstlisting}[title=Operazione 1, style=mysqlStyle]
-- inserimento di un nuovo prodotto nel magazzino --

DELIMITER |
CREATE PROCEDURE nuovoProdotto (IN codice INTEGER, IN nome VARCHAR(20), IN marca VARCHAR(20), IN quantita INTEGER, IN scadenza DATE, IN annata DATE, IN categoria VARCHAR(20), IN tipo VARCHAR(20), IN prezzo FLOAT, IN fornitore VARCHAR(11))
BEGIN
    START TRANSACTION;
    INSERT INTO prodotto (CodiceProdotto, NomeProdotto, Marca, Quantita, Annata, DataScadenza, Categoria, Tipo, Prezzo, IvaFornitore) VALUES
    (codice, nome, marca, quantita, annata, scadenza, categoria, tipo, prezzo, fornitore);
    COMMIT;
END |
DELIMITER ;
\end{lstlisting}

\begin{lstlisting}[title=Operazione 2, style=mysqlStyle]
--modificare la quantita di un prodotto --

DELIMITER |
CREATE PROCEDURE modificaQuantita (IN codiceProd INTEGER, IN quantita INTEGER)
BEGIN
    UPDATE prodotto AS P
    SET P.Quantita = P.Quantita + quantita
    WHERE P.CodiceProdotto=codice; 
END |
DELIMITER ;
\end{lstlisting}

\begin{lstlisting}[title=Operazione 3, style=mysqlStyle]
-- inserimento di una nuova prenotazione --

DELIMITER |
CREATE PROCEDURE nuovaPrenotazione (IN codice INTEGER, IN coperti INTEGER, IN nomeCliente VARCHAR(20), IN giorno DATE, IN ora DATE, IN tavolo INTEGER)
BEGIN
    START TRANSACTION;
    INSERT INTO prenotazione(CodicePrenotazione, NumeroCoperti, Cliente, Giorno, Ora, NumeroTavolo) VALUES
    (codice, coperti, nomeCliente, giorno, ora, tavolo);
    COMMIT;
END |
DELIMITER ;
\end{lstlisting}

\begin{lstlisting}[title=Operazione 4, style=mysqlStyle]
-- cancella prenotazione solo se la prenotazione e' successiva alla data corrente --

DELIMITER |
CREATE PROCEDURE eliminaPrenotazione (IN codice INTEGER)
BEGIN
    DELETE FROM prenotazione WHERE CodicePrenotazione=codice AND Giorno > CURDATE();
END |
DELIMITER ;
\end{lstlisting}

\begin{lstlisting}[title=Operazione 5, style=mysqlStyle]
-- inserisci nuovo dipendente nel personale --

DELIMITER |
CREATE PROCEDURE nuovoDipendente (IN cf VARCHAR(20), IN nome VARCHAR(20), IN cognome VARCHAR(20), IN giorno_libero VARCHAR(10), IN ruolo VARCHAR(20), IN stipendio FLOAT)
BEGIN
    INSERT INTO personale (CodiceFiscale, Nome, Cognome, GiornoLibero, Ruolo, Stipendio) VALUES
    (cf, nome, cognome, giorno_libero, ruolo, stipendio);
END |
DELIMITER ;
\end{lstlisting}

\begin{lstlisting}[title=Operazione 6, style=mysqlStyle]
-- modifica turno di lavoro --

DELIMITER |
CREATE PROCEDURE modificaTurno (IN dataTurno date, IN dipendente VARCHAR(20))
BEGIN
    UPDATE svolgimento_turno SET CodiceFiscale=dipendente
    WHERE DataTurno=dataTurno;
END |
DELIMITER ;
\end{lstlisting}

\subsection{Funzioni} %almeno 2 funzioni
\begin{lstlisting}[title=Operazione 11, style=mysqlStyle]
-- calcola il guadagno mensile dato dalla differenza tra la somma delle entrate e la somma delle uscite --

DELIMITER |
CREATE FUNCTION guadagnoTot (mese DATE) RETURNS FLOAT
BEGIN 
    DECLARE Guadagno FLOAT;
    DECLARE entrate FLOAT;
    DECLARE uscite FLOAT;
    
    SELECT SUM(U.Costo) INTO uscite
    FROM uscita AS U
    WHERE U.DataUscita >= '2019' + MONTH(mese) + '01' AND U.DataUscita <= '2019' + MONTH(mese) + '31';
    
    SELECT SUM(E.Costo) INTO entrate 
    FROM entrata AS E
    WHERE E.DataEntrata >= '2019' + MONTH(mese) + '01' AND E.DataEntrata <= '2019' + MONTH(mese) + '31';
    
    SET Guadagno = entrate-uscite;
    RETURN Guadagno;
END |
\end{lstlisting}

\begin{lstlisting}[title=Operazione 12, style=mysqlStyle]
-- trova il cameriere che effettua piu' ordini in media al giorno per decidere se promuoverlo --

CREATE FUNCTION promozione (mese DATE, cameriere VARCHAR(20)) RETURNS BOOL
BEGIN
    DECLARE ordini INTEGER;
    DECLARE promozione BOOL;
    
    SELECT COUNT(*) INTO ordini
    FROM ordine AS O
    WHERE O.CodiceCameriere=cameriere AND O.Giorno >= '2019' + MONTH(mese) + '01' AND O.giorno <= '2019' + MONTH(mese) + '31';
    
    IF ordini >= 20 THEN SET promozione=true;
    ELSE SET promozione=FALSE;
    END IF;
    
    RETURN promozione;
END |
\end{lstlisting}

\subsection{Trigger} %almeno 2 trigger
\begin{lstlisting}[title=Regola di vincolo 1, style=mysqlStyle]
-- un dipendente non puo' lavorare nel suo giorno libero--

CREATE TRIGGER noGiorniLiberi
BEFORE INSERT ON svolgimentoTurno
FOR EACH ROW
BEGIN
    DECLARE giornoLibero VARCHAR(10);
    
    SELECT P.GiornoLibero INTO giornoLibero
    FROM personale AS P;
    
    IF giornoLibero NOT IN (SELECT T.Giorno FROM personale AS P, svolgimentoTurno AS T WHERE P.CodiceFiscale=T.CodiceFiscale)
    THEN CALL modificaTurno(new.dataTurno, new.dipendente);
    END IF;
END |
DELIMITER ;
\end{lstlisting}

\begin{lstlisting}[title=Regola di vincolo 2, style=mysqlStyle]
-- un dipendente non puo' svolgere piu' di 6 turni a settimana --

DELIMITER |
CREATE TRIGGER MaxTurni
BEFORE INSERT ON svolgimentoTurno
FOR EACH ROW
BEGIN
    DECLARE turni INTEGER;
    
    SELECT COUNT(CodiceFiscale) INTO turni
    FROM svolgimentoTurno
    WHERE CodiceFiscale=new.CodiceFiscale;
    
    IF turni <= 6 THEN CALL inserisciTurno;
    END IF;
END |
DELIMITER ;
\end{lstlisting}
\begin{lstlisting}[title=Regola di vincolo 3, style=mysqlStyle]
-- un cliente non puo' prenotare un tavolo gia' prenotato --

DELIMITER |
CREATE TRIGGER tavoliPrenotati
BEFORE INSERT ON prenotazione
FOR EACH ROW
BEGIN
    IF new.tavolo NOT IN (SELECT P.tavolo FROM prenotazione AS P)
    THEN CALL inserisciPrenotazione;
    END IF;
END |
DELIMITER ;
\end{lstlisting}
