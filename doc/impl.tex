\section{Implementazione}
%devono contenere: descrizione testuale, codice, l'output(solo per query/procedure)
\subsection{Query} %almeno 6 query, di cui alcune possono essere procedure
\begin{lstlisting}[title=Operazione 9, style=mysqlStyle] 
-- stampa il nome, il codice e la quantita di tutti i cibi con data di scadenza a febbraio 2019 in ordine cresente di data --

CREATE VIEW CibiInScadenza as
SELECT DISTINCT C.CodiceProdotto, C.NomeProdotto, C.Quantita
FROM prodotto AS C
WHERE C.Categoria = 'Cibo' AND DataScadenza >= '2019-02-01' AND DataScadenza <= '2019-02-28'
ORDER BY DataScadenza;
\end{lstlisting}

\begin{lstlisting}[title=Operazione 10, style=mysqlStyle]
-- stampa il codice, il nome, la marca e l'annata di tutti i vini con annata risalente a massimo il 2008 e prezzo inferiore a 20 euro
-- ordinati per prezzo crescente --

CREATE VIEW viniEconomici as
SELECT DISTINCT V.CodiceProdotto, V.NomeProdotto, V.Marca, V.Annata
FROM prodotto AS V
WHERE V.Categoria='Bevanda' AND V.Tipo='Vino' AND V.Annata <= 2008 AND V.Prezzo < 20
ORDER BY Prezzo;
\end{lstlisting}

\begin{lstlisting}[title=Operazione 11, style=mysqlStyle]
-- stampa il numero dei tavoli liberi nel giorno 20/02/2019 con almeno 3 coperti 

CREATE VIEW tavoliLberi as
SELECT T.NumeroTavolo
FROM tavolo AS T
WHERE T.Coperti >= 3 AND T.NumeroTavolo NOT IN 
(SELECT P.NumeroTavolo FROM prenotazione AS P WHERE P.Giorno='2019-02-20');
\end{lstlisting}

\begin{lstlisting}[title=Operazione 12, style=mysqlStyle]
-- stampa nome e cognome di tutti i dipendenti di turno il lunedi --

CREATE VIEW dipendentiDiTurno as
SELECT DISTINCT P.Nome, P.Cognome
FROM personale AS P, svolgimentoTurno AS T
WHERE T.CodiceFiscale=P.CodiceFiscale AND T.Giorno = 'Lunedi';
\end{lstlisting}

\subsubsection{Procedure}
\begin{lstlisting}[title=Operazione 1, style=mysqlStyle]
-- operazione 1: inserimento nuovo prodotto --
DELIMITER |
CREATE PROCEDURE nuovoProdotto (IN codice INTEGER, IN nome VARCHAR(20), IN marca VARCHAR(20), IN quantita INTEGER, IN scadenza DATE, IN annata DATE, IN categoria VARCHAR(20), IN tipo VARCHAR(20), IN prezzo FLOAT, IN fornitore VARCHAR(11))
BEGIN
    START TRANSACTION;
    INSERT INTO prodotto (CodiceProdotto, NomeProdotto, Marca, Quantita, Annata, DataScadenza, Categoria, Tipo, Prezzo, IvaFornitore) VALUES
    (codice, nome, marca, quantita, annata, scadenza, categoria, tipo, prezzo, fornitore);
    COMMIT;
END |
DELIMITER ;
\end{lstlisting}

\begin{lstlisting}[title=Operazione 2, style=mysqlStyle]
-- operazione 2: cencellazione di un prodotto --
DELIMITER |
CREATE PROCEDURE eliminaProdotto (IN codice INTEGER)
BEGIN
    DELETE FROM prodotto WHERE CodiceProdotto=codice;
END |
DELIMITER ;
\end{lstlisting}

\begin{lstlisting}[title=Operazione 3, style=mysqlStyle]
--modificare la quantita di un prodotto --
DELIMITER |
CREATE PROCEDURE modificaQuantita (IN codice INTEGER, IN quantita INTEGER)
BEGIN
    UPDATE prodotto AS P
    SET P.Quantita = P.Quantita + quantita
    WHERE P.CodiceProdotto=codice;
END |
DELIMITER ;
\end{lstlisting}

\begin{lstlisting}[title=Operazione 4, style=mysqlStyle]
-- operazione 4: inserimento di una nuova prenotazione --
DELIMITER |
CREATE PROCEDURE nuovaPrenotazione (IN codice INTEGER, IN coperti INTEGER, IN nomeCliente VARCHAR(20), IN giorno DATE, IN ora DATE, IN tavolo INTEGER)
BEGIN
    START TRANSACTION;
    INSERT INTO prenotazione(CodicePrenotazione, NumeroCoperti, Cliente, Giorno, Ora, NumeroTavolo) VALUES
    (codice, coperti, nomeCliente, giorno, ora, tavolo);
    COMMIT;
END |
DELIMITER ;
\end{lstlisting}

\begin{lstlisting}[title=Operazione 5, style=mysqlStyle]
-- operazione 5: cancellazione di una prenotazione --
DELIMITER |
CREATE PROCEDURE eliminaPrenotazione (IN codice INTEGER)
BEGIN
    DELETE FROM prenotazione WHERE CodicePrenotazione=codice;
END |
DELIMITER ;
\end{lstlisting}

\begin{lstlisting}[title=Operazione 6, style=mysqlStyle]
-- operazione 6: inserisci nuovo dipendente --
DELIMITER |
CREATE PROCEDURE nuovoDipendente (IN cf VARCHAR(20), IN nome VARCHAR(20), IN cognome VARCHAR(20), IN giorno_libero VARCHAR(10), IN ruolo VARCHAR(20), IN stipendio FLOAT)
BEGIN
    INSERT INTO personale (CodiceFiscale, Nome, Cognome, GiornoLibero, Ruolo, Stipendio) VALUES
    (cf, nome, cognome, giorno_libero, ruolo, stipendio);
END |
DELIMITER ;
\end{lstlisting}

\begin{lstlisting}[title=Operazione 7, style=mysqlStyle]
-- operazione 7: licenzia dipendente --
DELIMITER |
CREATE PROCEDURE licenziaDipendente (IN dipendente VARCHAR(20))
BEGIN
    DELETE FROM personale WHERE CodiceFiscale=dipendente;
END |
DELIMITER ;
\end{lstlisting}

\begin{lstlisting}[title=Operazione 8, style=mysqlStyle]
    -- operazione 8: modifica turno di lavoro --
DELIMITER |
CREATE PROCEDURE modificaTurno (IN giorno VARCHAR(10), IN dipendente VARCHAR(20))
BEGIN
    UPDATE svolgimento_turno SET CodiceFiscale=dipendente
    WHERE Giorno=giorno;
END |
DELIMITER ;
\end{lstlisting}

\subsection{Funzioni} %almeno 2 funzioni
\begin{lstlisting}[title=Operazione 13, style=mysqlStyle]
-- calcola il guadagno mensile dato dalla differenza tra la somma delle entrate e la somma delle uscite --

DELIMITER |
CREATE FUNCTION guadagnoTot (mese DATE) RETURNS FLOAT
BEGIN 
    DECLARE Guadagno FLOAT;
    DECLARE entrate FLOAT;
    DECLARE uscite FLOAT;
    
    SELECT SUM(U.Costo) INTO uscite
    FROM uscita AS U
    WHERE U.DataUscita >= '2019' + MONTH(mese) + '01' AND U.DataUscita <= '2019' + MONTH(mese) + '31';
    
    SELECT SUM(E.Costo) INTO entrate 
    FROM entrata AS E
    WHERE E.DataEntrata >= '2019' + MONTH(mese) + '01' AND E.DataEntrata <= '2019' + MONTH(mese) + '31';
    
    SET Guadagno = entrate-uscite;
    
    RETURN Guadagno;
    
END |
\end{lstlisting}

\begin{lstlisting}[title=Operazione 14, style=mysqlStyle]
-- trova il cameriere che effettua piu ordini in media al giorno per decidere se promuoverlo o no --

CREATE FUNCTION promozione (mese DATE, cameriere VARCHAR(20)) RETURNS BOOL
BEGIN
    DECLARE ordini INTEGER;
    DECLARE promozione BOOL;
    
    SELECT COUNT(*) INTO ordini
    FROM ordine AS O
    WHERE O.CodiceCameriere=cameriere AND O.Giorno >= '2019' + MONTH(mese) + '01' AND O.giorno <= '2019' + MONTH(mese) + '31';
    
    IF ordini >= 20 THEN SET promozione=true;
    ELSE SET promozione=FALSE;
    END IF;
    
    RETURN promozione;
    
END |
\end{lstlisting}

\subsection{Trigger} %almeno 2 trigger
\begin{lstlisting}[title=Regola di vincolo 1, style=mysqlStyle]
-- un dipendente non puo lavorare nel suo giorno libero--
CREATE TRIGGER noGiorniLiberi
BEFORE INSERT ON svolgimentoTurno
FOR EACH ROW
BEGIN
    DECLARE giornoLibero VARCHAR(10);
    
    SELECT P.GiornoLibero INTO giornoLibero
    FROM personale AS P;
    
    IF giornoLibero NOT IN (SELECT T.Giorno FROM personale AS P, svolgimentoTurno AS T WHERE P.CodiceFiscale=T.CodiceFiscale)
    THEN CALL modificaTurno(new.giorno, new.dipendente);
    END IF;
END |
DELIMITER ;
\end{lstlisting}

\begin{lstlisting}[title=Regola di vincolo 2, style=mysqlStyle]
-- un dipendente non puo svolgere piu di 6 turni a settimana --
DELIMITER |
CREATE TRIGGER MaxTurni
BEFORE INSERT ON svolgimentoTurno
FOR EACH ROW
BEGIN
    DECLARE turni INTEGER;
    
    SELECT COUNT(CodiceFiscale) INTO turni
    FROM svolgimentoTurno
    WHERE CodiceFiscale=new.CodiceFiscale;
    
    IF turni <= 6 THEN CALL inserisciTurno;
    END IF;
    
END |
DELIMITER ;
\end{lstlisting}

\begin{lstlisting}[title=Regola di vincolo 3, style=mysqlStyle]
-- Un cliente non puo prenotare un tavolo gia prenotato. --
DELIMITER |
CREATE TRIGGER tavoliPrenotati
BEFORE INSERT ON prenotazione
FOR EACH ROW
BEGIN
    IF new.tavolo NOT IN (SELECT P.tavolo FROM prenotazione AS P)
    THEN CALL inserisciPrenotazione;
    END IF;
END |
DELIMITER ;
\end{lstlisting}
