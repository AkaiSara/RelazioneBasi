\section{Implementazione}


%\subsection{Tabelle} Le tabelle del database rispecchiano totalmente le relazioni dello schema logico-relazionale.
%Sono presenti, dunque, \textbf{numero} tabelle. La loro implementazione è presente nel file `'.sql` consegnato. %sostituisci numero con il numero effettivo delle tabelle

%devono contenere: descrizione testuale, codice, l'output(solo per query/procedure)

\subsection{Query} %almeno 6 query, di cui alcune possono essere procedure
\subsubsection{Procedure}

\begin{comment}
    operazione 3: stampa il nome e la quantità di tutti i cibi con data di scadenza al 10/12/2016
    opearzione 4: stampa la marca e il tipo di tutti i  vini o superalcolici, con annata risalente al 1987
    opearzione 5: stampa nome e PI di tutti i fornitori che vendono almeno 2 cibi con tipo "fresco"
    opearzione 6: stampa il codice e il costo di tutte le uscite con causale "acqua" in data 3/6/2015
    opearzione 7: stampa il codice e il costo di tutte le entrate del giorno 21/10/2016
    opearzione 8: stampa il codice e il numero di persone di tutti gli ordini serviti dalla cameriera "Anna" il giorno 15/04/2015 al tavolo "18" 
    operazione 9: stampa il numero di tutti i tavoli prenotati il giorno 31/12/2016 
    operazione 12: stampa il codice di tutte le prenotazioni per il tavolo numero 5 il giorno 24/08/2015
    operazione 16: stampa nome cognome di tutti i dipendenti di turno il lunedì
\end{comment}

\subsection{Funzioni} %almeno 2 funzioni

\begin{comment}
    operazione 18: calcola il guadagno mensile dato dalla differenza tra la somma delle entrate e la somma delle uscite
    operazione 19: trova il cameriere che effettua più ordini in media al giorno per decidere se promuoverlo o no
\end{comment}

\subsection{Trigger} %almeno 2 trigger

\begin{comment}
    operazione 1: inserimento di un nuovo prodotto
    operazione 2: cancellazione di un nuovo prodotto
    operazione 10: inserisci nuova prenotazione
    operazione 11: cancella prenotazione
    operazione 13: inserisci nuovo dipendente
    operazione 14: licenzia dipendente
    operazione 15: modifica lo stipendio del dipendente 
    operazione 17: modifica turno di lavoro
\end{comment}

